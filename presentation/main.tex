\documentclass{beamer}

% Theme and style
\usetheme{metropolis} % modern clean theme
\usepackage[ngerman]{babel}
\usepackage{graphicx}
\usepackage{listings}
\usepackage{caption}
\usepackage{subcaption}
\usepackage{amsmath}
\usepackage{eurosym}
\usepackage{siunitx}
\usepackage{tikz}
\usetikzlibrary{positioning, shapes, arrows.meta}

\sisetup{locale=DE, per-mode=symbol}
\DeclareSIUnit{\sieuro}{\mbox{\euro}}

%Hyphenation rules
\usepackage{hyphenat}
\hyphenation{Mathe-matik wieder-gewinnen}

% Metadata
\title{Einführung in die Simulation\\DESMO-J Projekt}
\subtitle{Ausfall von Maschinen}
\author{Schmidt L, Schlager A, Weilert A}
\institute{Paris Lodron Universität Salzburg}
\date{\today}

% Code style
\lstset{
    basicstyle=\ttfamily\small,
    backgroundcolor=\color{gray!10},
    frame=single,
    breaklines=true
}

\begin{document}

% Title slide
    \maketitle

% Overview
    \begin{frame}{Overview}
        \tableofcontents
    \end{frame}

% Section: Introduction
    \section{Einführung}
    \begin{frame}{Szenario}
        Eine Fabrik hat $m$ Maschinen und $s$ Service-MitarbeiterInnen.
        Die Maschinen brauchen immer wieder einen Service.
        In dieser Zeit verursachen die Maschinen Kosten, bis ein(e) MitarbeiterIn sie repariert hat.
        Sind alle MitarbeiterInnen beschäftigt,
        müssen die Maschinen auf eine Reparatur warten.
        Die Reparaturreihenfolge erfolgt nach dem FIFO Prinzip.\vspace*{1em}\newline
        \textit{Kosten}
        \begin{itemize}
            \item $\SI{10}{\sieuro\per\hour}$ pro MitarbeiterIn (immer)
            \item $\SI{50}{\sieuro\per\hour}$ pro Maschine (solange außer Betrieb)
        \end{itemize}
    \end{frame}

    \begin{frame}{Kosten}
        Die verursachten Kosten sind von der gesamten Dauer der inaktiven Phasen (\textbf{Downtime} DT) aller Maschinen und den
        Löhnen der MitarbeiterInnen abhängig.
        Sei die Anzahl der Maschinen fix, so ergibt sich die Kostenfunktion in Abhängigkeit von der Anzahl der MitarbeiterInnen,
        der Downtime und der gesamten Dauer der Simulation (\textbf{Total Time} TT).
        \[
            \mathrm{Cost}(s, \text{DT}, \text{TT}) = \text{DT}\,\cdot \SI{50}{\sieuro\per\hour} + \text{TT}\,\cdot \SI{10}{\sieuro\per\hour} \cdot s
        \]
    \end{frame}

    \begin{frame}{Ziel}
        Wir suchen die Anzahl $\hat{s}$ an MitarbeiterInnen, welche die Gesamtkosten für $m$ Maschinen minimiert.
        \[
             \hat{s} = \arg\min_{s \in \mathbb{N}} \mathrm{Cost}(s, \text{DT}, \text{TT})
        \]
        Aus den Ergebnissen für unterschiedliche $m$ soll außerdem ein Richtwert bestimmt werden, welcher
        die optimale Anzahl an MitarbeiterInnen pro Maschine beschreiben soll.
    \end{frame}

% Section: System Description
    \section{System Beschreibung}
    \begin{frame}{Simulation}
        Allgemeine Informationen zur Simulation
        \begin{itemize}
            \item Prozessorientierte Simulation mit Maschinen als Prozesse
            \item Simulationsdauer von 2 Arbeitsjahren (500 Tage, $\SI{8}{\hour}$)
            \item Zeitauflösung in Minuten
            \item $m\in\{1, \dots, 20\}$ und $s\in\{1,\dots, 20\}$
        \end{itemize}
    \end{frame}

    \begin{frame}{Wahrscheinlichkeitverteilungen}
        Breakdown Time Verteilung
        Repair Time Verteilung
    \end{frame}

    \begin{frame}{Factory Model}

    \end{frame}

    \begin{frame}{Maschinen Modell}
        Grafik
    \end{frame}

    \begin{frame}{MitarbeiterInnen Modell}
        Als Zahl die erhöht und verringert wird, je nach Verfügbarkeit von ArbeiterInnen.
        Falls keine MitarbeiterInnen verfügbar sind, dann kommt der Prozess in die Warteschlange.
    \end{frame}

% Section: Implementation
    \section{Implementation in Java}
    \begin{frame}[fragile]{Architecture}
        \begin{itemize}
            \item Object-oriented design
            \item Use of classes for events, entities, and the simulation environment
        \end{itemize}
    \end{frame}
    
    \begin{frame}{Factory Model}
        Factory Model Erklärung
    \end{frame}

    \begin{frame}{Machine Process}
        Machine Process Erklärung
    \end{frame}

    \begin{frame}{Main / Application}
        Main Erklärung
    \end{frame}

    \begin{frame}[fragile]{Example Code Snippet}
        \begin{lstlisting}[language=Java, caption=Basic event class]
public class MachineEvent extends Event {
    public void eventRoutine() {
        // Event logic here
    }
}
        \end{lstlisting}
    \end{frame}

% Section: Results
    \section{Ergebnisse}
    \begin{frame}{Ergebnisse}
        Für $m\in [1; 20]$ wurden jeweils $s\in[1;20]$ MitarbeiterInnen für zwei Arbeitsjahre simuliert.
        An den Ergebnissen ist die optimale Konfiguration zu entnehmen.
    \end{frame}

    \begin{frame}{Grafiken}
        20 Grafiken
    \end{frame}

    \begin{frame}{Interpretation}
        Slide mit optimalen Werte für jede Anzahl an Maschinen.
        Hier ist ein Muster zu erkennen.
        (Tabelle)
    \end{frame}

    \begin{frame}{Regression}
        Regressions grafik
    \end{frame}

% Section: Conclusion
    \section{Fazit}
    \begin{frame}{Zusammenfassung}
        \begin{itemize}
            \item Schwierigkeiten
            \item Entsprechen die Ergebnisse den Erwartungen
            \item Weitere Überlegungen
            \item Was war interessant?
        \end{itemize}
    \end{frame}

% Thank you
    \begin{frame}[standout]
        Vielen Dank!\\
        Fragen?
    \end{frame}

\end{document}
